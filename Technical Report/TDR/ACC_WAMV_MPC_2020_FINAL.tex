%%%%%%%%%%%%%%%%%%%%%%%%%%%%%%%%%%%%%%%%%%%%%%%%%%%%%%%%%%%%%%%%%%%%%%%%%%%%%%%%
%2345678901234567890123456789012345678901234567890123456789012345678901234567890
%        1         2         3         4         5         6         7         8

\documentclass[letterpaper, 10 pt, conference]{ieeeconf}  % Comment this line out if you need a4paper

%\documentclass[a4paper, 10pt, conference]{ieeeconf}      % Use this line for a4 paper

\IEEEoverridecommandlockouts                              % This command is only needed if 
                                                          % you want to use the \thanks command

\overrideIEEEmargins                                      % Needed to meet printer requirements.

%In case you encounter the following error:
%Error 1010 The PDF file may be corrupt (unable to open PDF file) OR
%Error 1000 An error occurred while parsing a contents stream. Unable to analyze the PDF file.
%This is a known problem with pdfLaTeX conversion filter. The file cannot be opened with acrobat reader
%Please use one of the alternatives below to circumvent this error by uncommenting one or the other
%\pdfobjcompresslevel=0
%\pdfminorversion=4

% See the \addtolength command later in the file to balance the column lengths
% on the last page of the document

% The following packages can be found on http:\\www.ctan.org
\usepackage{graphics}   % for pdf, bitmapped graphics files
\usepackage{graphicx}
\usepackage{epsfig}   % for postscript graphics files
\usepackage{mathptmx}   % assumes new font selection scheme installed
\usepackage{times}    % assumes new font selection scheme installed
\usepackage{amsmath}  % assumes amsmath package installed
\usepackage{amssymb}    % assumes amsmath package installed
\usepackage{siunitx}
\usepackage{multirow}
\usepackage{booktabs}
\usepackage{longtable}
\usepackage{rotating}
\usepackage{textcomp}
\usepackage{bm}
\usepackage{fancyhdr}
\usepackage{comment}
\usepackage{subfigure}

\title{\LARGE \bf
Model Predictive Control of a Catamaran Surface Vessel with a Functionally-defined Prediction Horizon
}


\author{Benjamin Armentor$^{1}$, Kyle Leleux$^{2}$, and Joshua Vaughan$^{3}$% <-this % stops a space
\thanks{$^{1}${\tt\small bma8468@louisiana.edu}}%
\thanks{$^{2}${\tt\small kxl0071@louisiana.edu}}%
\thanks{$^{3}$Joshua Vaughan is with the Department of Mechanical Engineering,
        University of Louisiana at Lafayette, Lafayette, LA 70504, USA
        {\tt\small joshua.vaughan@louisiana.edu}}%
}
% Attempting to add pre-submission header with this, but can't get it to show up, even with commenting out \thispagestyle{empty} and \pagestyle{empty}

%\lhead{\footnotesize{\textit{To be Submitted to:} ACC 2020 -- Due: 9/26/19}}
%\rhead{\footnotesize{\textit{Draft} -- \today}}

\bibliographystyle{IEEEtran}

\begin{document}
%\pagestyle{fancyplain}
\maketitle
\thispagestyle{empty}
%\pagestyle{empty}

\begin{abstract}
For mobile robots, such as Autonomous Surface Vehicles (ASVs), limiting the error from a target trajectory is necessary for effective and safe operation. However, environmental disturbances like wind, waves, and currents often make this difficult. This paper evaluates differences in performance between a Model Predictive Controller that uses a fixed prediction horizon and a Model Predictive Controller with a functionally-defined prediction horizon. The proposed algorithm changes the prediction horizon based on system states and measurements of environmental disturbances. The Model Predictive Controller performance is also compared to a PID-controlled system under the same disturbance conditions. The ASV used for this evaluation is a Wave-Adaptive Modular Vessel (WAM-V) from Marine Advanced Research, Inc. The functional prediction horizon resulted in a decrease in the IAE Euclidian-distance error from both the fixed-horizon controllers and the PID-controlled case in low-disturbance conditions. In sea state 3 conditions, the functional prediction horizon resulted in an increase in error from both the fixed-horizon controllers and the PID controller in all cases tested.
\end{abstract}

\section{INTRODUCTION}
A large research effort has been put forth in the field of autonomous vehicles on land \cite{Marquardt:14a, Guo:18a, Falcone:07a}, sea \cite{Fisher:10a, Miao:2018a, Kjerstad:10a}, and air \cite{Gupte:12a, Shim:05a, Bellingham:02a, Mahe:18a}. While many people, even outside the scientific community, are aware of this push for self-driving automobiles and unmanned aerial drones, some are unaware that self-driving boats and submarines are already in use and under continued development. Engineers from both industry and the research community alike are attempting to further automate boats, including small/mid-sized worker class boats, commonly used for oceanographic surveying and defense \cite{Fisher:10a}, large-scale shipping vessels, and micro-scale vessels for hobbyists \cite{Miao:2018a}. Many of these autonomous surface vessels (ASVs) are controlled by cascaded PID controllers \cite{Fisher:10a, Kjerstad:10a, Klinger:17a, Sarda:16a}. While PID control provides an easy, well-understood control method, it does not explicitly account for changes in environmental disturbances like wind, waves, or currents, and therefore is subject to high trajectory tracking error when operating in these conditions.

Controllers that do include compensation for environmental disturbances is an active area of research \cite{Das:15a,Perez:11a,Fossen:11c}. Optimal controllers, which take the objective and the dynamics of the system into account, are one way to include these environmental disturbances in the controller because, with models of these disturbances, the forces and moments produced by these disturbances can be estimated. One such optimal control method is Model Predictive Control (MPC), a control strategy that requires solving a non-convex optimization problem at each time step in order to minimize a cost function over $n$ future time steps. The first input generated by the optimization is applied to the system at the current time step, then the solution procedure is repeated at each subsequent time step.

MPC has been used extensively for process control and power transmission control systems, but has been, until recently, limited by the computational speeds necessary for application to other types of systems \cite{Qin:03a, Morari:99a, Garcia:89a}. With improvements in processing power, MPC has been applied to drone control \cite{Mahe:18a, Falanga:18a}, line-of-sight path following \cite{Oh:10a}, and incorporated with neural networks for mobile robots \cite{Li:16a}. Choosing the size of the prediction horizon, the number of time steps ahead to predict over and solve the optimal control problem, is a key component of MPC. The horizon must be small enough to allow solution of the non-convex optimization problem in real time, but large enough, relative to the bandwidth of the system, to anticipate future disturbances and changes in the desired trajectory and calculate the necessary corrective action. This paper proposes a functionally-defined horizon size for ASVs operating in the presence of wind and wave disturbances.

The next section introduces the system being evaluated, including a general vessel overview and equations of motion. The simulation procedure and analysis are discussed in Section III. Then, Section IV discusses the performance of a PID-controlled ASV. This performance is compared to a traditional MPC controller and a MPC controller with a functionally-defined prediction horizon in Section V.
%
\section{SYSTEM DESCRIPTION}
The Wave-Adaptive Modular Vessel (WAM-V) Unmanned Surface Vessel (USV) from Marine Advanced Research, Inc., shown in Figure \ref{fig:WAM-V}, is a 4.85 $m$ catamaran-style boat with inflatable pontoons, two engine pods, and a payload tray supported by an articulating suspension system. Some of the system characteristics are listed in Table \ref{Physical Characteristics}. The WAM-V is actuated by two Torqeedo Cruise 2.0 motors mounted at the stern of the vessel capable of azimuth angles of \textpm45\ensuremath{^\circ} from the centerline.
%
\begin{figure}[tb]
  \centering
  \vspace{0.3in}
  \includegraphics[width=\linewidth]{Figures/WAM_V}
  \caption{Wave-Adaptive Modular Vessel (WAM-V)}
  \label{fig:WAM-V}
\end{figure}
%
\begin{table}[tb]
\caption{WAM-V Physical Characteristics}
\vspace{-0.1in}
\label{Physical Characteristics}
  \begin{center}
    \begin{tabular}{lc}
    \textbf{Item} & \textbf{Value}\\
    Mass & 180 $kg$\\
    Moment of Inertia & 250 $kgm^2$\\
    Overall Length & 4.85 $m$\\
    Waterline Length & 4.17 $m$\\
    LCG & 1.3 $m$\\
    Beam & 2.44 $m$\\
    Center-to-Center Beam & 2.012 $m$\\
    Draft & 0.1 $m$\\
    Lateral Projected Area & 2.018 $m^2$\\
    Frontal Projected Area & 1.248 $m^2$\\
    \end{tabular}
  \end{center}
\end{table}

For this paper, the planar model shown in Figure \ref{fig:Coordinates} was adopted to model the WAM-V dynamics. The $North$, $N$, and $East$, $E$, axes are an inertial coordinate system pointing in the corresponding cardinal directions. The $X_B$--$Y_B$ axes are body-fixed to the WAM-V and are rotated from the inertial reference frame by the angle $\psi$.
\begin{figure}[tb]
  \centering
  \includegraphics[width=\linewidth]{Figures/WAMV_Coordinates_Planar.pdf}
  \caption{Planar Model of the WAM-V}
  \label{fig:Coordinates}
\end{figure}
%

The equations of motion describing the dynamics of the WAM-V are \cite{Fossen:11c}:
%
\begin{equation}
\bm{M}\dot{\bm{\nu}} + \bm{C}(\bm{\nu})\bm{\nu} + \bm{D}(\bm{\nu})\bm{\nu} = \bm{\tau} + \bm{\tau}_{wind} + \bm{\tau}_{wave}
\end{equation}
%
\begin{equation}
\dot{\bm{\eta}} =
\left[
\begin{matrix}
\dot{x} & \dot{y} & \dot\psi
\end{matrix}
\right]^T
\end{equation}
%
\begin{equation}
\bm{\nu} = \left[
\begin{matrix}
u & v & r
\end{matrix}
\right]^T
\end{equation}
%
where $\dot{\bm{\eta}}$ is used to describe the vessel's North, $\dot{x}$, and East, $\dot{y}$, linear velocities and the angular velocity around the Z-axis, $\dot{\psi}$, in the Earth-fixed reference frame. The vessel's surge velocity, $u$, sway velocity, $v$, and yaw rate, $r$, in the body-fixed frame are described by $\bm{\nu}$. The forces and moments produced by the propulsion system of the WAM-V and wind and wave disturbances are captured by $\bm{\tau}$, $\bm{\tau}_{wind}$, and $\bm{\tau}_{wave}$, respectively. The rotation matrix from the body-fixed frame to the Earth-fixed frame is a simple rotation about the Z-axis.

The inertia matrix, $\bm{M}$, is the sum of a rigid-body mass matrix, $\bm{M}_{RB}$, and an added-mass matrix, $\bm{M}_{AM}$:
%
\begin{equation}
\bm{M} = \bm{M}_{RB} + \bm{M}_{AM} =
\left[
\begin{matrix}
m-X_{\dot{u}} & 0 & -my_G\\
0 & m-Y_{\dot{v}} & mx_G - Y_{\dot{r}}\\
-my_G & mx_G - N_{\dot{v}} & I_z - N_{\dot{r}}
\end{matrix}
\right]
\end{equation}
%
where the terms $X_{\dot{u}}$, $Y_{\dot{v}}$, $Y_{\dot{r}}$, $N_{\dot{v}}$, and $N_{\dot{r}}$ are hydrodynamic terms following the SNAME (1950) notation \cite{SNAME:50a}, $m$ is the mass of the vessel, $x_G$ and $y_G$ are the distances between the center of mass and the origin of the body-fixed frame, and $I_z$ is the moment of inertia about the Z-axis of the body-fixed frame. If the center of mass is chosen as the origin of the body-fixed frame, $x_G$ and $y_G$ are zero. The Coriolis matrix, $\bm{C}\left(\bm{\nu}\right)$, also consists of separate rigid-body, $\bm{C}_{RB}$, and added-mass, $\bm{C}_{AM}$, matrices:
%
\begin{equation}
\bm{C}\left(\bm{\nu}\right) = \bm{C}_{RB} + \bm{C}_{AM}
\end{equation}
%
where $\bm{C}_{RB}$ and $\bm{C}_{AM}$ are:
%
\begin{align}
\bm{C}_{RB} &=
\left[
\begin{matrix}
0 & 0 & -m\left(x_Gr + v\right)\\
0 & 0 & -m\left(y_Gr - u\right)\\
m\left(x_Gr + v\right) & m\left(y_Gr - u\right) & 0
\end{matrix}
\right]\\
%
\bm{C}_{AM} &=
\left[
\begin{matrix}
0 & 0 & Y_{\dot{v}}v + r\frac{\left(Y_{\dot{r}} + N_{\dot{v}}\right)}{2}\\
0 & 0 & -X_{\dot{u}}\\
-Y_{\dot{v}}v - r\frac{\left(Y_{\dot{r}} + N_{\dot{v}}\right)}{2} & X_{\dot{u}} & 0
\end{matrix}
\right]
\end{align}
%
The drag matrix, $\bm{D}\left(\bm{\nu}\right)$, contains both linear, $\bm{D}_{lin}$, and non-linear, $\bm{D}_{nonlin}$, components, such that:
%
\begin{equation}
\bm{D}\left(\bm{\nu}\right) = \bm{D}_{lin} + \bm{D}_{nonlin}
\end{equation}
%
\begin{align}
\bm{D}_{lin} &=
\left[
\begin{matrix}
X_u & 0 & 0 \\
0 & Y_v & Y_r \\
0 & N_v & N_r
\end{matrix}
\right]\\
%
\bm{D}_{nonlin} &= \left[
\begin{matrix}
X_{uu}|u| & 0 & 0\\
0 & Y_{vv}|v| + Y_{vr}|r| & Y_{rv}|v| + Y_{rr}|r|\\
0 & N_{vv}|v| + N_{vr}|r| & N_{rv}|v| + N_{rr}|r|
\end{matrix}
\right]
\end{align}
%
The linear drag terms, $X_u$, $Y_v$, $Y_r$, $N_v$, and $N_r$, represent the drag force or moment produced in the corresponding equation of motion due to motion in the subscripted direction. Similarly, the nonlinear drag terms, $X_{uu}$, $Y_{vv}$, $Y_{vr}$, $Y_{rv}$, $Y_{rr}$, $N_{vv}$, $N_{vr}$, $N_{rv}$, and $N_{rr}$, represent the forces and moments produced in the first subscripted direction due to motion in the second subscripted direction.

Disturbances obviously affect the motion of the WAM-V. The main disturbances are current, waves, and wind. If the current is assumed to be uniform, it can be modeled via the relative velocity between the WAM-V and the water. The relative velocity affects the added-mass, Coriolis, and drag terms. The equation of motion is altered by separating the rigid body and hydrodynamic terms by the WAM-V's total velocity and velocity relative to the water, $\nu_r$:
%
\begin{equation}
\bm{\nu}_{r} = \bm{\nu} - \bm{\nu}_{current}
\end{equation}
%
\begin{align}
\bm{M}_{RB}\dot{\bm{\nu}} &+ \bm{C}_{RB}\left(\bm{\nu}\right)\bm{\nu} + \bm{M}_{AM}\dot{\bm{\nu}}_{r} + \bm{C}_{AM}\left(\bm{\nu}_{r}\right)\bm{\nu}_{r} \\
&+\bm{D}\left(\bm{\nu}_r\right)\bm{\nu}_r = \bm{\tau} + \bm{\tau}_{wind} + \bm{\tau}_{wave} \nonumber
\end{align}
%
where $\dot{\bm{\nu}}_{r}$ is the relative acceleration of the vessel relative to the water, and $\bm{\nu}_{current}$ is the velocity of the current flow. The structure of the Coriolis matrices does not change, but the forces and moments produced by the added-mass matrix are dependent upon the relative velocity of the vessel rather than the total velocity. Since the linear drag terms are a function of the velocity, they do change \cite{Klinger:17a}. The nonlinear drag matrix elements, $X_{uu}|u|$, $Y_{vv}|v| + Y_{vr}|r|$, $Y_{rv}|v| + Y_{rr}|r|$, $N_{vv}|v| + N_{vr}|r|$, and $N_{rv}|v| + N_{rr}|r|$, change because they are multiplied by the absolute value of a velocity.

The control inputs to the system, $\tau$, are defined as:
\begin{equation}
\bm{\tau} = \left[
\begin{matrix}
T_{p}\cos\left(\phi_{p}\right) + T_{s}\cos\left(\phi_{s}\right)\\
\\
T_{p}\sin\left(\phi_{p}\right) + T_{s}\sin\left(\phi_{s}\right)\\
\\
\left(\frac{B}{2}\right)\left(T_{p}\cos\left(\phi_{p}\right) - T_{s}\cos\left(\phi_{s}\right)\right) -\\ LCG\left(T_{p}\sin\left(\phi_{p}\right) + T_{s}\sin\left(\phi_{s}\right)\right)
\end{matrix}
\right]
\end{equation}
%
where $T_p$ and $T_s$ are the port and starboard thrust, respectively, $\phi_p$ and $\phi_s$ are the azimuth angles of the port and starboard thrusters, respectively, $B$ is the center-to-center beam width of the WAM-V, and $LCG$ is the perpendicular distance from the thrusters to the center of mass.
%
\section{SYSTEM SIMULATION}

Simulations of the system with a forward velocity of 3.0~$\frac{m}{s}$ and following a sinusoidal heading command with an amplitude of 135\ensuremath{^\circ} and period of 30 seconds were performed. Tracking this trajectory can be difficult for non-holonomic vessels when subjected to environmental disturbances, particularly those acting perpendicular to the desired surge direction. These environmental disturbances included wind and wind-produced waves. The wind was simulated as having a variable magnitude and direction that is normally distributed based on data found in \cite{Joffre:87a}. Wind forces and moments were modeled according to \cite{Fossen:11c}:
%
\begin{equation}
\bm{\tau}_{wind} =
\frac{1}{2}\rho_{air}V_{wind_r}^2
\left[
\begin{matrix}
C_x(\gamma_{wind_r})A_{Fw}\\
C_y(\gamma_{wind_r})A_{Lw}\\
C_n(\gamma_{wind_r})A_{Lw}L
\end{matrix}
\right]
\end{equation}
%
where $V_{wind_r}$ is the wind's velocity relative to the vessel, $\gamma_{wind_r}$ is the relative angle of attack of the wind with respect to the vessel, $\rho_{air}$ is the density of air, and $A_{fw}$ and $A_{lw}$ are the frontal and lateral projected areas of the vessel, respectively. Wind coefficients, $C_x$, $C_y$, and $C_n$, can be found in a table for moving vessels \cite{Fossen:11c}. The range of simulated wind speeds spanned from 0.001--7.64 $\frac{m}{s}$, the highest of which is the wind speed required to produce the maximum significant wave height of sea state 3 on the Modified Pierson-Moscowitz (MPM) spectrum.  This was chosen as the upper limit based on the maximum operational conditions of the WAM-V. Zero is excluded from the range in order to have non-zero wave heights. These high-frequency waves have a low spectral energy density, so their contribution to the overall vessel motion is negligible.

The wave field is composed of a single repeating wave with a fixed direction and wavelength. The wavelength was selected to match a sample wave from the Gulf of Mexico \cite{Liu:12a}. Wave forces and moments were approximated for the planar model as waves acting on a block-shaped ship, modified for a catamaran-style vessel \cite{Fossen:11c}:
%
\begin{equation}
X_{wave} = \rho_w g \left(2B_{hull}\right)L_{wl}T\cos\left(\beta_{wave_r}\right)s
\end{equation}
\begin{equation}
Y_{wave} = \rho_w g B_{hull}L_{wl}T\sin\left(\beta_{wave_r}\right)s
\end{equation}
\begin{equation}
N_{wave} = \frac{1}{24}\rho_w g B_{hull}L_{wl}\left(L_{wl}^2 - B_{hull}^2\right)\sin\left(\beta_{wave_r}\right)s^2
\end{equation}
%
where $\rho_w$ is the density of water, $g$ is the acceleration due to gravity, $B_{hull}$ is the beam width of one pontoon hull, $L_{wl}$ is the submerged length of the WAM-V, $\beta_{wave_r}$ is the relative angle of attack of the waves, and $s$ is the current slope of the wave. The original equations in \cite{Fossen:11c} assume a solid hull, but for the catamaran-style vessel, the wave sway force, $Y_{wave}$, and yaw moment, $N_{wave}$, are only acting on one hull at a time, so only the wetted-area for one hull is used. Although the WAM-V is equipped with a suspension system between its pontoons and payload tray, these forces were not included.

\section{PID CONTROL}
The WAM-V system response was simulated across the full range of disturbances when utilizing PID control. Separate PID controllers for speed and heading were created and tuned for the low-disturbance case.
%
\begin{figure}[tb]
\begin{center}
\subfigure[Low-Disturbance Environment]
{
    \includegraphics[width=\linewidth]{Figures/PlanarPath_Beta_90_Vw_0p001.pdf}
    \label{fig:PlanarPath_V0}
}
\subfigure[Sea State 3 Environment]
{
    \includegraphics[width=\linewidth]{Figures/PlanarPath_Beta_90_Vw_7p64.pdf}
    \label{fig:PlanarPath_V7p6}
}
\vspace{-0.125in}
\caption{WAM-V Motion in Low-Disturbance and Sea-State 3 Environments}
\label{fig:PlanarPath}
\end{center}
\vspace{-0.2in}
\end{figure}
%

An aerial view of the WAM-V's path over time, starting at the origin, can be seen in Figure \ref{fig:PlanarPath}. In both the low-disturbance and sea state 3 case, the WAM-V travels a similar shape as the desired trajectory, but there is error in the sway induced from the Coriolis effects of the angular velocity. This is because there is no direct control of sway velocity. The Euclidian distance from the desired trajectory over time is shown in Figure \ref{fig:EOT}. The PID controller outperforms the fixed-horizon MPC controllers, which will be presented in Section V, in the low-disturbance case, but not in the sea state 3 case. The PID controller's tracking performance degrades as the environmental disturbances increase, but the MPC controllers maintain nearly the same trajectory across the range of disturbances tested here.
\begin{figure}[!b]
\begin{center}
\subfigure[Low-Disturbance Environment]
{
    \includegraphics[width=\linewidth]{Figures/EOT_Beta_90_Vw_0p001.pdf}
    \label{fig:EOT_v0}
}
\subfigure[Sea State 3 Environment]
{
    \includegraphics[width=\linewidth]{Figures/EOT_Beta_90_Vw_7p64.pdf}
    \label{fig:EOT_v7p6}
}
\vspace{-0.125in}
\caption{Euclidian Distance from Desired Trajectory in Low-Disturbance and Sea-State 3 Environments}
\label{fig:EOT}
\end{center}
\vspace{-0.2in}
\end{figure}
%

\section{MODEL PREDICTIVE CONTROL}
The cost function of MPC controller is defined as:
%
\begin{align}
\label{cost}
Cost &= \int_0^p \left(\frac{1}{2}\bm{\chi}^T\bm{Q}\bm{\chi} - \left(\sqrt{\bm{g_w}} \cdot \bm{\upsilon}^*\right)^T\bm{\upsilon}\right)dt\\
\bm{c} &= \left[T_p, T_s, \phi_p, \phi_s\right]^T\\
\bm{\chi} &= \left[\left[\bm{R_{\psi}}^{-1}\bm{\eta}\right], \left[\bm{\nu}\right], \left[\bm{c}\right]\right]\\
\bm{\upsilon} &= \left[\left[\bm{R}_{\psi}^{-1}\bm{\eta}\right], \left[\bm{\nu}\right]\right]
\end{align}
%

where $p$ is the number of steps in the prediction horizon, $\bm{\chi}$ is the concatenated vector of body-fixed states and actuator inputs, $\bm{Q}$ is a positive-definite matrix of weights on the body-fixed states and actuator inputs, $\bm{g_w}$ are the weights from the diagonal of $\bm{Q}$ on only the body-fixed states, $\bm{\upsilon}^*$ is a vector of desired body-fixed states, $\bm{\upsilon}$ is the vector of body-fixed states with no control inputs, and $\bm{R}_{\psi}$ is the rotation matrix for a simple rotation about the $Z$-axis. The system weights and penalty on actuation effort used in the simulations presented in this paper are listed in Table \ref{tab:mpc_weights}.
%
\subsection{Fixed Horizon}
%
\begin{table}[t!]
\caption{MPC Controller Weights}
\vspace{-0.1in}
\label{tab:mpc_weights}
  \begin{center}
    \begin{tabular}{cc}
    \textbf{State} & \textbf{Value}\\
    $x_B$ & 10^{-12}\\
    \vspace{-0.075in}
    \\
    \vspace{-0.075in}
    $y_B$ & $\frac{\pi}{180}$\\
    \\
    $\psi$ & 1.0\\
    \vspace{-0.075in}
    \\
    $u$ & 1.0\\
    \vspace{-0.075in}
    \\
    $v$ & $\frac{\pi}{180}$\\
    \vspace{-0.075in}
    \\
    $r$ & 1.0\\
    \vspace{-0.075in}
    \\
    Actuators & 10^{-3}
    \end{tabular}
  \end{center}
\end{table}
%
The model predictive controllers were constructed to use 0.5, 1, and 1.5 seconds as the nominal lookahead times. In the low-disturbance case, the 0.5-second horizon controller resulted in a 40.7\% increase in Integral Absolute Error (IAE), where error is defined as the  Euclidian distance from the desired trajectory. The 1-second and 1.5-second horizon controllers begin tracking with error measures comparable to the PID controller, but, as seen in Figure \ref{fig:PlanarPath_V0}, after cresting the top of the "Figure-8" path, the responses track the desired trajectory with approximately 10.3\% more error than the PID controller in the low-disturbance case. When subjected to the same sea state 3 conditions as the PID-controlled vessel, the controllers perform similarly, as shown in Figure \ref{fig:PlanarPath_V7p6}. In the low-disturbance case, the MPC controllers experienced approximately 40.7\%, 10\%, and 10.5\% more error than the PID controller for the nominal lookahead times of 0.5, 1, and 1.5 seconds, respectively. In the sea state 3 disturbance case, the 0.5-second horizon controller performed 20\% worse than the PID controller, but the 1-second and 1.5-second horizon controllers improved error by 7.3\% and 8.8\%, respectively. The IAE errors for the PID, fixed-horizon, and adaptive-horizon controllers as a function of wind speed is summarized in Figure \ref{fig:IAE_total}.

\begin{figure}[tb]
  \centering
  \includegraphics[width=\linewidth]{Figures/TOTAL_IAE.pdf}
  \caption{IAE Error as a Function of Wind Speed}
  \label{fig:IAE_total}
\end{figure}

\subsection{Functionally-defined Horizon}

The change in performance of the fixed horizon controllers was compared to a functionally-defined prediction horizon based on the system states and environmental disturbances. The adaptive horizon is defined as:
%
\begin{align}
\label{horizon}
p = p^* &+ \frac{1}{t_s}\left(|u| + \frac{|r|}{10\left(\frac{\pi}{180}\right)} + 4H_{sig} + \frac{|\beta_{wave_{r}}|}{\frac{\pi}{2}}\right)\\
&+ \frac{|v|}{|v_{desired}| + v_{tol}} + \frac{V_{wind_r}}{V_{wind_{max}}} + \frac{|\gamma_{wind_r}|}{\frac{\pi}{2}} \nonumber
\end{align}
%
where $p^*$ is the nominal prediction horizon size, $t_s$ is the sampling time of the controller, $H_{sig}$ is the significant wave height, $V_{wind_{max}}$ is the wind speed to produce waves of 1.25 meters, and $v_{tol}$ is a tolerable amount of sway velocity specified. This value was chosen as 0.05 $\frac{m}{s}$. For the WAM-V, $v_{desired}$ is zero because sway is an uncontrollable DOF for non-holonomic vessels. This term increases the prediction horizon size when the difference between $v$ and $v_{desired}$ is greater than this tolerable level. The output of $\left(\ref{horizon}\right)$ is rounded to the nearest integer. The body-fixed velocity terms are included based on the decay rate from hydrodynamic drag. The wave forces and moments were found to have a greater impact on the system response, so these terms were given higher weights. As the coefficients on all of the terms after $p^*$ in $\left(\ref{horizon}\right)$ are increased, the MPC controller approaches an infinite-horizon controller. Though the upper limit of $p$ is still governed by the time required to solve the optimization problem online, no limit is currently enforced.

The adaptive horizon controller was simulated using wind speeds of 0.001, 3.42, 5.92, and 7.64 $\frac{m}{s}$, which represent the wind speed boundaries of sea states 0--3. The horizon was re-evaluated after every 2 seconds. This time interval and the weights on the system states were chosen based on the velocity decay rate of the vessel from maximum surge, sway, and angular velocities. Table \ref{tab:times} contains the times for the WAM-V to decay to 5\% of steady-state for the different velocities.
%The surge velocity reaches 5\% of steady-state faster at higher speeds because the nonlinear drag term, $X_{uu}$, is quadratic in $u$ \cite{Sarda:16a}. The sway velocity takes the longest to settle to 5\% of steady-state because of coupling between sway and yaw motion. Similarly, the yaw velocity requires nearly the same time to settle as the sway velocity when the actuators produce both sway forces and yaw moments.
Combining the magnitude of these drag forces at maximum conditions relative to the magnitude of the actuators guided the selection of the weights of the terms in the adaptive horizon equation.
%
\begin{table}[tb]
\vspace{0.1in}
\caption{Velocity Decay Times}
\vspace{-0.1in}
\label{tab:times}
  \begin{center}
    \begin{tabular}{lccc}
    \textbf{State} & $\bm{u_{max}}$ & $\bm{v_{max}}$ & $\bm{r_{max}}$\\
    u & 5.54 $s$ & 5.18 $s$ & 9.45 $s$\\
    v & -- & 8.27 $s$ & 10.24 $s$\\
    r & -- & 8.19 $s$ & 3.75 $s$\\
    \end{tabular}
  \end{center}
\end{table}

As shown in Figure \ref{fig:EOT}, the adaptive horizon controller maintained similar or less Euclidian-distance error from the desired trajectory than both the 0.5-second and 1.5-second fixed horizon controllers in the low-disturbance case, but nearly matched the error of the 0.5-second controller for the sea state 3 case. In the low-disturbance case, the adaptive horizon controller improves performance by approximately 32\%, 13\%, 12.8\%, and 13.3\% from the PID, 0.5-second, 1-second, and 1.5-second horizon controllers respectively. However, in the sea state 3 case, the adaptive horizon controller resulted in 21\%, 0.6\%, 31\%, and 33\% increases in error. This is likely because the optimal control problem can be redefined every two seconds, so the resulting inputs to the system are not as smooth as the fixed-horizon controllers. This is demonstrated in the thrust differences between a fixed-horizon and the adaptive horizon controllers, shown in Figure \ref{fig:thrust}.

\begin{figure}[b]
  \centering
  \includegraphics[width=\linewidth]{Figures/Thrusts_Beta_90_Vw_0p001.pdf}
  \caption{Thrusts for 1.5s Fixed-Horizon Controller and Adaptive Horizon Controller in Low Disturbances}
  \label{fig:thrust}
\end{figure}

The ITAE and RMS errors for the full range of wind speeds tested for the controllers are shown in Figures \ref{fig:RMS} and \ref{fig:ITAE}. In both error measures, the PID controller outperforms the fixed-horizon controllers across the most of the tested wind speeds except 7.0 $\frac{m}{s}$ and above. Both the ITAE and RMS errors increase as the wind speed increases. Based on both the RMS and ITAE error measures, the adaptive horizon performs similar to a PID controller at low wind speeds. However, because there is more error later in the trajectory, this controller performs worse than all controllers based on the ITAE error metric when operating in high disturbance conditions.

Based on average time to compute and store one optimal input sequence, the time required to solve the optimization problem when using the adaptive horizon increased by approximately 450\% for the low-disturbance case and 900\% for the sea state 3 case. This is because the controller's optimization window changed from 1.5 seconds to 7--17 seconds, depending on the system states and environmental disturbances, as shown in Figure \ref{fig:HOT}. 
\begin{figure}[tb]
  \centering
  \includegraphics[width=\linewidth]{Figures/TRAJ_RMS_Beta_90.pdf}
  \caption{RMS Error as a Function of Wind Speed}
  \label{fig:RMS}
\end{figure}
\begin{figure}[tb]
  \centering
  \includegraphics[width=\linewidth]{Figures/TRAJ_ITAE_Beta_90.pdf}
  \caption{ITAE Error as a Function of Wind Speed}
  \label{fig:ITAE}
\end{figure}
\begin{figure}[tb]
  \centering
  \includegraphics[width=\linewidth]{Figures/Adapt_HorizonSize_Beta_90.pdf}
  \caption{Adaptive Horizon Size in Low-Disturbance and Sea-State 3 Environments}
  \label{fig:HOT}
\end{figure}
\section{Conclusion}
This paper compared the performance of a fixed-horizon Model Predictive Controller and a Model Predictive Controller where the lookahead time was functionally determined by the system states and disturbance measurements. The performance was evaluated using a model Wave-Adaptive Modular Vessel (WAM-V) from Marine Advanced Research, Inc. It was shown that the total trajectory tracking error increased when using the adaptive horizon controller except in the low-disturbance case. Additionally, there was a large increase in optimization time. Furthermore, the adaptive horizon outperforms the fixed-horizon controllers and PID controller only in cases with very minimal disturbances. As wind speeds and wave heights increased, the adaptive horizon window increased from about 7 seconds to about 17 seconds. Based on the Euclidian distance error over time, the 1-second and 1.5-second-horizon controllers outperformed the PID controller in the sea state 3 disturbance case. However, these controllers almost always perform worse than the PID controller based on the RMS and ITAE error measures.

%\addtolength{\textheight}{-12cm}   % This command serves to balance the column lengths
                                  % on the last page of the document manually. It shortens
                                  % the textheight of the last page by a suitable amount.
                                  % This command does not take effect until the next page
                                  % so it should come on the page before the last. Make
                                  % sure that you do not shorten the textheight too much.
%\section*{APPENDIX}

\section*{Acknowledgments}
The authors would like to thank the Louisiana Board of Regents and L3ASV of L3Harris Corporation for the funding of this research.
\bibliography{ACC-WAMV_MPC-2020.bib}

\end{document}
